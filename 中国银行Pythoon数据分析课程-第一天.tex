\documentclass[UTF8]{ctexart}
\usepackage[a4paper]{geometry}
\usepackage{xcolor}
\usepackage{graphicx}
\usepackage{accsupp}	
\newcommand{\emptyaccsupp}[1]{\BeginAccSupp{ActualText={}}#1\EndAccSupp{}}
\usepackage{listings}
\title{中国银行Python数据分析课程-第一天}
\author{作者:江浩}
\definecolor{commentcolor}{RGB}{85,139,78}
\definecolor{stringcolor}{RGB}{206,145,108}
\definecolor{keywordcolor}{RGB}{34,34,250}
\definecolor{backcolor}{RGB}{220,220,220}

\lstset{						%高亮代码设置
	language=python, 					%Python语法高亮
	linewidth=0.9\linewidth,      		%列表list宽度
	%basicstyle=\ttfamily,				%tt无法显示空格
	commentstyle=\color{commentcolor},	%注释颜色
	keywordstyle=\color{keywordcolor},	%关键词颜色
	stringstyle=\color{stringcolor},	%字符串颜色
	%showspaces=true,					%显示空格
	numbers=left,						%行数显示在左侧
	numberstyle=\tiny\emptyaccsupp,		%行数数字格式
	numbersep=5pt,						%数字间隔
	frame=single,						%加框
	framerule=0pt,						%不划线
	escapeinside=@@,					%逃逸标志
	emptylines=1,						%
	xleftmargin=3em,					%list左边距
	backgroundcolor=\color{backcolor},	%列表背景色
	tabsize=4,							%制表符长度为4个字符
	gobble=4							%忽略每行代码前4个字符
}

\date{\today}
\begin{document}
\maketitle
\begin{figure}[ht]
	\centering
	\includegraphics[scale=1.0]{images/python-logo.png}
\end{figure}
\tableofcontents

\section{Python的安装}
	\subsection{下载}
	\begin{enumerate}
		\item 下载地址
		\item 安装步骤
	\end{enumerate}
\section{Python的启动运行}
\section{Python的脚本文件模式}
\begin{lstlisting}
	import pandas as pd 
	print("as")
\end{lstlisting}
教学目标:
1.	掌握Python环境搭建,常用的IDE环境使用。
2.	掌握模块的安装和导入,pip安装时候的重定向镜像主机参数使用。
3.	掌握使用pandas创建和读取xlsx,csv等常用表格文件。
4.	掌握几种简单的数据类型。
课程内容:
		1.Python介绍环境安装
		2.Jupyter notebook及插件安装配置和启动等常规操作
		3.pandas读写xlsx文件。
		4.DataFrame初步认识。
		5.整形、浮点型、字符串、字典类型等数据类型掌握。
课程实践:
		本课实践内容通过pandas实现xlsx,csv读取并对表格数据进行修改,然后保存到新的表格文件中。该实践内容报表自动化生成任务、拆分表格,报表生成中会有广泛的应用。


\section{Python打开文件Excel文件}

\end{document}